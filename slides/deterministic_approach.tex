\begin{frame} 
	\frametitle{Speed estimation from position measurements} 
	Consider a car and a (quite inaccurate) model of it
	\begin{columns}[onlytextwidth]
		\column{.4\textwidth}
		\begin{figure}[h]
			\includegraphics[width=\textwidth]{fig/auto_real}%
			\caption*{The car we want to model.\footnotemark} 
		\end{figure}
		\column{.1\textwidth}
		\begin{tikzpicture}
		\draw [myarrows] (0, 0) -- (1, 0);
		\end{tikzpicture}
		\column{.4\textwidth}
		\begin{block}{The (quite inaccurate) model}
		Position at time $t$: $(\xi(t), \zeta(t))$ \\%{\footnotesize ($x$ and $y$  will be used later with different a meaning)}\\
		Equations of motion:
		\begin{align*}
		  \ddot{\xi}(t)& = u_1(t) \\
		  \ddot{\zeta}(t)& = u_2(t)
		\end{align*}
		where 
		\begin{itemize}
			%\item $M$ mass of the car\\
			\item $u_1(t)$ acceleration along first coordinate, i.e. $\xi$\\
			\item $u_2(t)$ acceleration along scond coordinate, i.e. $\zeta$ 
		\end{itemize}
		%These are the control inputs.
		\end{block}
	\end{columns}
	
	
	\footnotetext{\href{https://commons.wikimedia.org/wiki/File\%3A2015_BMW_i8_(20039281571)_(2).jpg}{\tiny{Image by Edvvc from London, UK (2015 BMW i8)}}}
\end{frame}

\begin{frame} 
	\frametitle{Model rewriting}
	We want to rewrite the model in a form that better higlights the input, output, and state of the system
    \begin{columns}[onlytextwidth]
    \column{.25\textwidth}
        \begin{align*}
        \ddot{\xi}(t)& = u_1(t) \\
        \ddot{\zeta}(t)& = u_2(t)
        \end{align*}
    \column{.1\textwidth}
    \begin{tikzpicture}
        \draw [myarrows] (0, 0) -- (1, 0);
    \end{tikzpicture}
    \column{.8\textwidth}
    	\begin{align*}
    	  \mat{\dot{\xi}\\\ddot{\xi}\\\dot{\zeta}\\\ddot{\zeta}}&=
          \mat{0&1&0&0\\0&0&0&0\\0&0&0&1\\0&0&0&0}\mat{\xi\\\dot{\xi}\\{\zeta}\\\dot{\zeta}} + \mat{0&0\\1  &0\\0&0 \\0&1} \mat{u_1(t)\\u_2(t)}\\
          \mat{y_1(t)\\y_2(t)}&=\mat{1&0&0&0\\0&0&1&0}\mat{\xi\\\dot{\xi}\\{\zeta}\\\dot{\zeta}}
    	\end{align*}
    \end{columns}
\end{frame}

\begin{frame} 
    \frametitle{Model rewriting}
	We want to rewrite the model in a form that better higlights the input, output, and state of the system
    \begin{columns}[onlytextwidth]
        \column{.2\textwidth}
        \begin{align*}
        \ddot{\xi}(t)& = u_1(t) \\
        \ddot{\zeta}(t)& = u_2(t)
        \end{align*}
        \vspace*{2em}
        \column{.1\textwidth}
        \begin{center}
           \begin{tikzpicture}
           \draw [myarrows] (0, 0) -- (1, 0);
           \end{tikzpicture} 
        \end{center}
        \vspace*{1em}
        \column{.6\textwidth}
        \begin{align*}
        \mat{\dot{\xi}\\\ddot{\xi}\\\dot{\zeta}\\\ddot{\zeta}} 
        &=
        \mat{0&1&0&0\\0&0&0&0\\0&0&0&1\\0&0&0&0}
        \tikz[baseline]{
            \node[fill=blue!10,anchor=base,rounded corners] (state)
            {$\mat{{\xi}\\\dot{\xi}\\{\zeta}\\\dot{\zeta}}$};
        }
        + 
        \mat{0&0\\1 &0\\0&0 \\0&1} 
        \tikz[baseline]{
            \node[fill=blue!10,anchor=base,rounded corners] (input)
            {$\mat{u_1(t)\\u_2(t)}$};
        }\\
        \tikz[baseline]{
            \node[fill=blue!10,anchor=base,rounded corners] (output)
            {$\mat{y_1(t)\\y_2(t)}$};
        }
        &=\mat{1&0&0&0\\0&0&1&0}\mat{\xi\\\dot{\xi}\\{\zeta}\\\dot{\zeta}}
        \end{align*}
        
        
        \begin{minipage}{5cm}\hfill\end{minipage}
        \begin{minipage}{3cm}
        \tikz[na]\node (statedef) {$\bm{x}(t)$: state};\vspace{-0.8em} 
        \tikz[na]\node (inputdef) {$\bm{u}(t)$: input};\vspace{-0.8em}
        \tikz[na]\node (outputdef) {$\bm{y}(t)$: output}; 
        \end{minipage}        
    \end{columns}
    \begin{tikzpicture}[overlay]
        \path[->](state) edge (statedef);
        \path[->](input) edge [out=-90, in=0] (inputdef);
        \path[->](output) edge [out=-90, in=-180] (outputdef);
    \end{tikzpicture}
\end{frame}

\begin{frame} 
    \frametitle{State space form}
    The names $A$, $B$, $C$ are often used in state space form. We can write 
    \begin{align*}
    \dot{\bm{x}}(t) &= A \bm{x}(t) + B \bm{u}(t)\\
    \bm{y}(t) &= C \bm{x} (t)
    \end{align*}
    \begin{columns}    %[onlytextwidth,t]
    \column{.4\textwidth}
        where 
        \begin{itemize}
            \item ${\bm{x}}(t)=\mat{{\xi}\\\dot{\xi}\\{\zeta}\\\dot{\zeta}}$ state of the system
            \item ${\bm{y}}(t)=\mat{y_1(t)\\y_2(t)}$ measured output
            \item ${\bm{u}}(t)=\mat{u_1(t)\\u_2(t)}$ input
        \end{itemize}
    \column{.5\textwidth}
    A graphical representation can be given as
    %\begin{block}
        \begin{tikzpicture}[auto,  >=latex']
        \node [input, name=input] {};
        \node [block, right = 3em of input] (blockB) {$B$};
        \node [sum, right of=blockB] (sum) {};
        \node [input, right of=sum] (dstate) {$\frac{1}{s}$};
        \node [block, right of=dstate] (integrator) {$\frac{1}{s}$};
        \node [input, right of=integrator] (state) {};
        \node [block, right of=state] (blockC) {$C$};
        \node [block, below of=integrator] (blockA) {$A$};
        \node [output, right = 3em of blockC] (output) {};
        
        \draw [draw,->] (input) -- node {$\bm{u}(t)$} (blockB);
        \draw [->] (blockB) -- node {} (sum);
        \draw [->] (sum) -- node {$\dot{\bm{x}}(t)$} (integrator);
        \draw [->] (integrator) -- node {$\bm{x}(t)$} (blockC);
        \draw [->] (blockC) -- node [name=y] {  $\bm{y}(t)$}(output);
        \draw [->] (state) |- node {}(blockA);
        \draw [->] (blockA) -| node[pos=0.99] {$+$} 
        node [near end] {} (sum);
        \end{tikzpicture}
    %\end{block}
    
    Simplified form
    
         \begin{tikzpicture}[auto,  >=latex']
         \node [input, name=input] {};
         \node [block, right = 3em of input] (blockAB) {$A$, $B$};
         \node [input, right of=blockAB] (state) {$x(t)$};
         \node [block, right of=state] (blockC) {$C$};
         \node [output, right = 3em of blockC] (output) {$y(t)$};
         
         \draw [draw,->] (input) -- node {$\bm{u}(t)$} (blockAB);
         \draw [->] (blockAB) -- node {$\bm{x}(t)$} (blockC);
         \draw [->] (blockC) -- node [name=y] {$\bm{y}(t)$}(output);
         \end{tikzpicture}
    \end{columns}
\end{frame}

\begin{frame} 
    \frametitle{Conversion to discrete time}
    Our estimator will be as a discrete time system. We need to convert our original model 
    
    \begin{itemize}
        \item $\Delta t$  sampling time interval
        \item $t_0$ initial time stamp
        \item ${\bm{x}}(k)$ is the $k^{th}$ samples of the state, i.e., $\bm{x}(k)=\bm{x}(t_0+k\Delta t)$. Similar notation for input and output
        \item $A_d$: discrete time version of the matrix $A$
        \item $B_d$: discrete time version of the matrix $B$
    \end{itemize}
     
    \begin{columns}[onlytextwidth]
        \column{.4\textwidth}
        \begin{block}{Continuous time}
            \vspace*{-1em}
            \begin{align*}
            \dot{\bm{x}}(t) &= A \bm{x}(t) + B \bm{u}(t)\\
            \bm{y}(t) &= C \bm{x}(t) 
            \end{align*}
        \end{block}    
        \column{.1\textwidth}
        \begin{tikzpicture}
        \draw [myarrows] (0, 0) -- (1, 0);
        \end{tikzpicture}
        \column{.4\textwidth}
        \begin{block}{Discrete time}
            \vspace*{-1em}
            \begin{align*}
            \bm{x}(k+1) &= A_d \bm{x}(k) + B_d \bm{u}(k)\\
            \bm{y}(k) &= C \bm{x}(k) 
            \end{align*}
        \end{block}
    \end{columns}
\end{frame}

\begin{frame} 
	\frametitle{System matrices in discrete time}
	\begin{columns}[onlytextwidth]
		\column{.4\textwidth}
		\begin{block}{Continuous time}
			\vspace*{-1em}
			\begin{align*}
			\dot{\bm{x}}(t) &= A \bm{x}(t) + B \bm{u}(t)\\
			\bm{y}(t) &= C \bm{x}(t) 
			\end{align*}
			\vspace*{-2em}
			\begin{align*}
			A&=
			\mat{0&1&0&0\\0&0&0&0\\0&0&0&1\\0&0&0&0}
			\\
			B&=\mat{0&0\\ 1 &0\\0&0 \\0&1}
			\end{align*}
		\end{block}
		\column{.2\textwidth}
		\centering
		\begin{tikzpicture}
		\draw [myarrows] (0, 0) -- (1, 0);
		\end{tikzpicture}
		\column{.4\textwidth}
		\begin{block}{Discrete time \footnotemark}
			\vspace*{-1em}
			\begin{align*}
			\bm{x}(k+1) &= A_d \bm{x}(k) + B_d \bm{u}(k)\\
			\bm{y}(k) &= C \bm{x}(k) 
			\end{align*}
			\vspace*{-2em}
			\begin{align*}
			A_d&=\mat{1&\Delta t&0&0\\0&1&0&0\\0&0&1&\Delta t\\0&0&0&1}
			\\
			B_d&=\mat{\frac{\Delta t^2}{2}&0\\ \Delta t &0\\0&\frac{\Delta t^2}{2} \\0&\Delta t}
			\end{align*}
		\end{block}
	\end{columns}
\footnotetext{\href{http://users.isy.liu.se/rt/fredrik/edu/sensorfusion/lecture9.pdf}{\small{For the discretization step see e.g., G. Fredrik. \emph{Sensor fusion} (slides)}}}
\end{frame}

\begin{frame}
    \frametitle{Estimation approaches}
    We consider two approaches
    \begin{itemize}
        \item [1.] Speed could be estimated by numerical differentiation from the measured position (Numerical differentiation)
        \item [2.] We could estimate the \emph{whole} state of the system and then, extract the speed of the car from the estimated state (Observer approach)
    \end{itemize}

     \begin{columns}[t]
        \column{.4\textwidth}
        \begin{block}{1. Numerical differentiation}
            \vspace*{-1em}
            \begin{align*}
            \hat{\dot{\xi}}(k)& = \frac{y_1(k) - y_1(k-1)}{\Delta t}\\
            \hat{\dot{\zeta}}(k)& = \frac{y_2(k) - y_2(k-1)}{\Delta t}
            \end{align*}
        \end{block}
        \column{.6\textwidth}
        \begin{block}{2. Observer}
        	\begin{enumerate}
        		\item [1.] Estimate the \emph{whole} state of the system, $\hat{\bm{x}}(k)$
        	
            	\item [2.] Extract speed information from the second and fourth elements of the state, i.e. 
            	$\hat{\dot{\xi}}(k)= \hat{x}_1(k)$, 
            	 $\hat{\dot{\zeta}}(k)=\hat{x}_3(k)$
        	\end{enumerate}
        \end{block}	
    \end{columns} 
\end{frame}

\begin{frame}
    \frametitle{Observer derivation}
    \begin{columns}
    	\column{.4\textwidth}
			Assume 
			\begin{itemize}
			  \item Model and parameters of the system are known
			  \item Input and output are measured without noise
			  \item An initial estimate of the state is known $\hat{\bm{x}}(0)$ 
			\end{itemize}
			\vspace{1em}
			We could define the observer as\vspace{-0.8em}
			\begin{align*}
				\hat{\bm{x}}(k+1) &= A_d \hat{\bm{x}}(k) + B_d \bm{u}(k)\\
				\hat{\bm{y}}(k) &= C\hat{\bm{x}}(k) 
			\end{align*}
		\column{.4\textwidth}
		\begin{block}{Diagram}
			\hfill \tikz[na]\node (label1) {True model};\vspace{0.8em}	
			\begin{tikzpicture}[auto,  >=latex']
				\node [input, name=input] {};
				\node [input, name=connection, right of=input]{};
				\node [block, right of=connection] (blockAB) {$A_d$, $B_d$};
				\node [input, right of=blockAB] (state) {$x(k)$};
				\node [block, right of=state] (blockC) {$C$};
				\node [output, right = 4em of blockC] (output) {$y(k)$};
				\begin{scope}[on background layer]
				\draw[fill=blue!30, rounded corners] ($(blockAB.north west)+(-0.1,0.4)$)  rectangle ($(blockC.south east)+(0.2,-0.2)$);
				\end{scope}
				
				\draw [draw,->] (input) -- node {$u(k)$} (blockAB);
				\draw [->] (blockAB) -- node {$\bm{x}(k)$} (blockC);
				\draw [->] (blockC) -- node [name=y] {$\bm{y}(k)$}(output);
				
				\node [block, below = 3em of blockAB] (blockObserverAB) {$A_d$, $B_d$};
				\node [block, below = 3em of blockC] (blockObserverC) {$C$};
				\node [output, right = 4em of blockObserverC] (outputObserver) {};
				\begin{scope}[on background layer]
				\draw[fill=red!30, rounded corners] ($(blockObserverAB.north west)+(-0.1,0.3)$)  rectangle ($(blockObserverC.south east)+(0.2,-0.2)$);
				\end{scope}
				
				\draw [draw,->] (connection) |- node {} (blockObserverAB);
				\draw [draw,->] (blockObserverAB) -- node {$\hat{\bm{x}}(k)$} (blockObserverC);
				\draw [->] (blockObserverC) -- node [name=y] {$\hat{\bm{y}}(k)$}(outputObserver);
				
				%\path[->](Observer) edge (label1);
			\end{tikzpicture}
			
			\hfill \tikz[na]\node (label2) {Observer};
			
		\end{block}
    \end{columns}
\end{frame}

\begin{frame}
	\frametitle{Open loop observer - Example}
	Trends of velocities are well tracked. Input is known and system parameters are known.
	 \begin{figure}[b]
	 	\includegraphics[width=0.8\textwidth]{fig/observer_ex_0}
	 	\caption*{}
	 \end{figure}
 	 
 	 \insertimptext{\bf{Initial errors on speed and position are not corrected}}
\end{frame}

\begin{frame}
	\frametitle{A possible solution: Luenberger observer}
	Key idea: use the measured variable for correcting the estimated state
	\vspace{1em}
	
	New equations of the observer
	\vspace{-1em}
	
	\begin{minipage}{0.6\textwidth}
		\begin{align*}
		\hat{\bm{x}}(k+1) &= A_d \hat{\bm{x}}(k) + B_d \bm{u}(k) + 
		\tikz[baseline]{
			\node[fill=blue!10,anchor=base,rounded corners] (corr_term)
			{$L(\bm{y}(k)-\hat{\bm{y}}(k))$};}\\
		\hat{\bm{y}}(k) &= C\hat{\bm{x}}(k) 
		\end{align*}
	\end{minipage}
	\begin{minipage}{0.3\textwidth}
		\tikz[na]\node (descr) {Correcting term}; 
	\end{minipage}
	\begin{tikzpicture}[overlay]
	\path[->](corr_term) edge (descr);
	\end{tikzpicture}\\[2em] 
	\pause
	Consider now the error on the estimated state: $\tilde{\bm{x}}(k)=\bm{x}(k)-\hat{\bm{x}}(k)$. We can write\\[-1.5em]
	\begin{align*}
	\hat{\bm{x}}(k+1)-\bm{x}(k+1) &= A_d \hat{\bm{x}}(k) + B_d \bm{u}(k) -  
	A_d \hat{\bm{x}}(k) - B_d \bm{u}(k) - L(\bm{y}(k)-\hat{\bm{y}}(k))\\
	\tilde{\bm{x}}(k+1) &= A_d \hat{\bm{x}}(k)- LC\bm{x}(k) -  
	A_d \hat{\bm{x}}(k) + LC\hat{\bm{y}}(k)\\
	\tilde{\bm{x}}(k+1) &= (A_d - LC)\bm{x}(k)-(A_d-LC) \hat{\bm{x}}(k) \\ 
	\tilde{\bm{x}}(k+1)&=(A_d - LC)\tilde{\bm{x}}(k1) 
	\end{align*}
	
	\pause
	The equation can be rewritten as\hfill \\
	$\qquad\tilde{\bm{x}}(k)=(A_d-LC)^k\tilde{\bm{x}}(0)$
\end{frame}

\begin{frame}
    \frametitle{Luenberger observer}
    Key idea: use the error in the measured variable for correcting the estimated state
    \vspace{1em}
    
    New equations of the observer
    \vspace{-1em}
    
    \begin{minipage}{0.6\textwidth}
        \begin{align*}
        \hat{\bm{x}}(k+1) &= A_d \hat{\bm{x}}(k) + B_d \bm{u}(k) + 
            \tikz[baseline]{
            \node[fill=blue!10,anchor=base,rounded corners] (corr_term)
            {$L(\bm{y}(k)-\hat{\bm{y}}(k))$};}\\
        \hat{\bm{y}}(k) &= C\hat{\bm{x}}(k) 
        \end{align*}
    \end{minipage}
    \begin{minipage}{0.3\textwidth}
        \tikz[na]\node (descr) {Correcting term}; 
    \end{minipage}
    \begin{tikzpicture}[overlay]
        \path[->](corr_term) edge (descr);
    \end{tikzpicture}\\[2em] 
    Consider now the error on the estimated state: $\tilde{\bm{x}}(k)=\bm{x}(k)-\hat{\bm{x}}(k)$. We can write\\[-1.5em]
    \begin{align*}
        \hat{\bm{x}}(k+1)-\bm{x}(k+1) &= A_d \hat{\bm{x}}(k) + B_d \bm{u}(k) -  
        A_d \hat{\bm{x}}(k) - B_d \bm{u}(k) - L(\bm{y}(k)-\hat{\bm{y}}(k))\\
        \tilde{\bm{x}}(k+1) &= A_d \hat{\bm{x}}(k)- LC\bm{x}(k) -  
        A_d \hat{\bm{x}}(k) + LC\hat{\bm{y}}(k)\\
        \tilde{\bm{x}}(k+1) &= (A_d - LC)\bm{x}(k)-(A_d-LC) \hat{\bm{x}}(k) \\ 
        \tilde{\bm{x}}(k+1)&= \tikz[baseline]{\node[fill=blue!10,anchor=base,rounded corners](dyn_err){$(A_d - LC)$};}\tilde{\bm{x}}(k1) \tikz[baseline]{\node[fill=blue!10,anchor=base,rounded corners](no_input){+ \phantom{x}};}
    \end{align*}

	The equation can be rewritten as\hfill \tikz[na]\node (no_input_comment) {No dependency on input};\\
	$\qquad\tilde{\bm{x}}(k)=(A_d-LC)^k\tilde{\bm{x}}(0)$\hfill \tikz[na]\node (comment) {Dynamics of the state estimation error};
	\begin{tikzpicture}[overlay]
		\path[->] (no_input_comment) edge (no_input);
		\path[->] (comment) edge (dyn_err);
	\end{tikzpicture}
\end{frame}

\begin{frame}
	\frametitle{Exponential decay - scalar case}
	We focus on the simplified case, where the state is a scalar: $\tilde{x}(k+1)= (a_d-lc)\tilde{x}(k)$
	
	%In this case %$x(k)=(a_d-lc)^k
	\begin{itemize}
		%\item The value of the error at time $k=0$ is unknown
		\item If $|a_d-lc|<1$, then it is guaranteed that $|\tilde{x}(k+1)< |\tilde{x}(k)|$
		\item As $k$ grows, it is guaranteed that $|\tilde{x}(k)|$ decreases towards 0
	\end{itemize}

	The non-scalar case requires all eigenvalues of $A_d-LC$ to have absolute value smaller than 1.
	
	\begin{figure}[b]
		\includegraphics[width=0.8\textwidth]{fig/exponential_decay}
		\caption*{Exponential decay for different values of $a_d-lc$.}
	\end{figure}
	
	
	%\insertimptext{Convergency is guranteed for all values of $\tilde{\bm{x}}(0)$}
	
\end{frame}

\begin{frame}
    \frametitle{Example}
    True and estimated position of the car, along with the true and estimated speed
    %\movie{\includegraphics[width=0.8\textwidth]{fig/observer_ex_1}}{video/observer_ex_1.avi}
    \begin{figure}
    	\includegraphics[width=0.9\textwidth]{fig/observer_ex_1}
    \end{figure}
\end{frame}

\begin{frame}
	\frametitle{Estimation with position measurement noise}
	$\xi(k)$ coordinate is affected by noise with uniform distribution between -2 and 2\\
	$\zeta(k)$ coordinate is affected by noise with Gaussian distribution with mean 0 and standard deviation 1
	\begin{figure}
		\includegraphics[width=0.9\textwidth]{fig/observer_ex_2}
	\end{figure}
\end{frame}

\begin{frame}
	\frametitle{Estimation with position and acceleration measurement noise}
	$u_1(k)$ measurements are affected by noise with uniform distribution between -2 and 2\\
	$u_2(k)$ measurements are affected by noise with Gaussian distribution with mean 0 and standard deviation 1
	\begin{figure}
		\includegraphics[width=0.9\textwidth]{fig/observer_ex_3}
	\end{figure}
\end{frame}

\begin{frame}
	\frametitle{Summary}
	\begin{itemize}
		\setlength\itemsep{1.5em}
		\item Numerical difference is a viable approach only when measurement noise is negligible
		\item Using feedback from the output can compensate initial estimation errors
		\item Definition of the observer requires knowledge of the system parameters
		\item Multiple gains of the observer can be used as long the eigenvalues of $A_d-LC$ have all absolute value smaller than 1
	\end{itemize}

\insertimptext{\bf{What would it be the best gain?}}
\end{frame}